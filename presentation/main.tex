\documentclass[polish,aspectratio=169]{beamer}

% wide screen
% \documentclass[aspectratio=169]{beamer}


%%% YOUR PACKAGES HERE %%%
\usepackage{comment}
\usepackage{hyperref}
\usepackage{biblatex}
\addbibresource{bibliography/bibliography.bib}
\usepackage{subfigure}
\usepackage{graphicx}
\usepackage{geometry}
\usepackage{float}
\usepackage{listings}
\usepackage{xcolor}

% polish language
\usepackage[polish]{babel}
\usepackage{polski}



%%% IMPORT PG PRESENTATION STYLE %%%
\include{pgbeamer/pgbeamer}


%%% YOUR OPTIONS HERE %%%

\title[Analiza algorytmów dla dominowania rzymskiego słabo spójnego]{Analiza algorytmów dla dominowania rzymskiego słabo spójnego}
\subtitle{Analysis of algorithms for weakly connected Roman domination}
\author{inż. Paulina Brzęcka}
\date{\today}

\setbeamercovered{invisible}
\begin{document}

\pgtitleframe

\begin{frame}{Definicja problemu}
    \begin{block}{Definicja}
        Funkcję dominującą rzymską słabo spójną (WCRDF) na grafie \( G \) definiuje się jako taką funkcję dominującą rzymską \( f \colon V(G) \to \{0, 1, 2\} \), dla której zbiór wierzchołków
        \[
        \{ u \in V(G) : f(u) \in \{1, 2\} \}
        \]
        stanowi jednocześnie słabo spójny zbiór dominujący.
        
        \vspace{1em}

        Wagę funkcji \( f \) definiuje się jako:
        \[
        f(V) = \sum_{u \in V} f(u)
        \]

        Liczbą dominowania rzymskiego słabo spójnego grafu \( G \) nazywamy najmniejszą możliwą wagę funkcji \( f \) spełniającej powyższe warunki i oznaczamy ją symbolem:
        \[
        \gamma^{\mathrm{wc}}_R(G)
        \]
    \end{block}
\end{frame}

\begin{frame}{Cel pracy}
    \begin{itemize}
        \item analiza algorytmów dla dominowania rzymskiego słabo spójnego,
        \item opisanie już istniejących rozwiązań i opracowanie własnych,
        \item analiza i porównanie ich skuteczności,
        \item znalezienie możliwych praktycznych zastosowań.
    \end{itemize}
\end{frame}

\begin{frame}{Pytania badawcze}
    \begin{itemize}
        \item Jakie algorytmy są w stanie znaleźć WCRDF? Które z nich są w stanie znaleźć dodatkowo najmniejszą sumę wag WCRDF?
        \item Czy czas i jakość działania algorytmów będzie uzależniony od klasy grafów?
        \item Czy i jakie algorytmy heurystyczne mogą skutecznie przybliżyć wartość liczby dominowania rzymskiego słabo spójnego w czasie krótszym niż dokładne algorytmy?
        \item Czy hiperparametry algorytmu mrówkowego można dostroić w taki sposób, aby ten algorytm znajdował liczbę dominowania rzymskiego słabo spójnego bliską optymalnej?
    \end{itemize}
\end{frame}

\begin{frame}{Algorytmy}
    \begin{table}[h!]
        \centering
        \begin{tabular}{|l|l|}
        \hline
        \textbf{Algorytm} & \textbf{Złożoność czasowa} \\
        \hline
        brute familyorce & $O(3^n \cdot n^2)$ \\
        \hline
        liniowy dla drzew & $O(n)$ \\
        \hline
        programowania liniowego 1 & wykładnicza \\
        \hline
        programowania liniowego 2 & wykładnicza \\
        \hline
        mrówkowy & $O(num\_iterations \cdot num\_ants \cdot n^2)$ \\
        \hline
        aproksymacyjny & wykładnicza \\
        \hline
        zachłanny & $O(n^2)$ - grafy rzadkie, $O(n^3)$ - grafy gęste \\
        \hline
        \end{tabular}
        \caption{Porównanie złożoności czasowej różnych klas algorytmów}
        \label{tab:algorithms_complexity}
    \end{table}

\end{frame}

\begin{frame}{Wyniki}
    \begin{figure}
        \centering
        \includegraphics[width=1\textwidth]{images/trees.png}
        \caption{Porównanie czasów działania algorytmów - grafy drzewiaste.}
    \end{figure}    
\end{frame}

\begin{frame}{Wyniki}
    \begin{figure}
        \centering
        \includegraphics[width=0.9\textwidth]{images/image1.png}
        \caption{Wybrane wyniki dla drzew.}
    \end{figure}    
\end{frame}

\begin{frame}{Wyniki}
    \begin{figure}
        \centering
        \includegraphics[width=0.8\textwidth]{images/alorithms.png}
        \caption{Porównanie algorytmów przybliżonych.}
    \end{figure}    
\end{frame}

\begin{frame}{Zastosowania praktyczne}
    \begin{figure}
        \centering
        \includegraphics[width=1\textwidth]{images/image.png}
        \caption{Rozmieszczenie zabezpieczeń sieci energetycznych.}
    \end{figure}    
\end{frame}

\begin{frame}{Wnioski}
    \begin{itemize}
        \item udało się odpowiedzieć na pytania badawcze,
        \item konieczność  doboru właściwego algorytmu do grafu lub klasy grafu,
        \item słaba jakość rozwiązań algorytmu mrówkowego,
        \item pokazanie potencjalnych rozwiązań praktycznych w ujęciu teoretycznym.
    \end{itemize}
\end{frame}

\begin{frame}{Dalsze kierunki badań}
    \begin{itemize}
        \item implementację i testy potencjalnych ulepszeń dla algorytmu zachłannego,
        \item rozszerzenie testów na inne klasy, jak i na inne wielkości grafów,
        \item podniesienie jakości wyników algorytmu mrówkowego, między innymi poprzez inną implementację heurystyki lokalnej oraz strategii feromonowej
        \item opracowanie algorytmów dokładnych, rozwiązywalnych w czasie wielomianowym dla innych klas grafów
        \item weryfikację i przełożenie teoretycznych rozważań na temat praktycznych zastosowań tego problemu na praktyczną analizę i realizację

    \end{itemize}
\end{frame}

% \setbeamercovered{transparent}
% \begin{frame}{Bibliografia}
%     \nocite{*}
%     \printbibliography
% \end{frame}

\pglastframe

\end{document}