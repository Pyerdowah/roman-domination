\chapter{Wstęp i cel pracy}
Tematem pracy jest analiza algorytmów znajdujących funkcję dominującą rzymską słabospójną w grafach.
Problem znajdywania liczby dominowania rzymskiego słabospójnego jest problemem NP-trudnym. Wersja decyzyjna tego problemu jest NP-zupełna. Nie istnieją zatem dokładne algorytmy rozwiązujące problem w czasie wielomianowym. Dlatego niniejsza praca dokonuje analizy dostępnych i proponowanych algorytmów rozwiązujących ten problem w sposób zarówno dokładny, jak i przybliżony, w celu znalezienia możliwie skutecznych rozwiązań oraz zastosowań.

\section{Cel pracy}
Celem pracy jest analiza algorytmów dla dominowania rzymskiego słabo spójnego, w tym opisanie już istniejących rozwiązań oraz opracowanie własnych, porównanie ich skuteczności oraz możliwych praktycznych zastosowań.

\section{Zakres pracy}
W ramach pracy dokonano systematycznego przeglądu literatury. W literaturze proponowano wiele algorytmów dokładnych o czasie wykładniczym, między innymi algorytmy wykorzystujace programowanie liniowe. Dodatkowo, w wielu publikacjach skupiono się na algorytmach dla konkretnych klas grafów. W literaturze zostały również zdefiniowane algorytmy niedokładne, aproksymacyjne, o różnej jakości rozwiązania.
Na podstawie znalezionej literatury zaimplementowane zostały dwa algorytmy programowania liniowego oraz $2(1+\epsilon)(1 + \ln(\Delta - 1))$-aproksymacyjny. W ramach własnej pracy, zaimplementowano algorytm dokładny brute force, zachłanny, liniowy dokładny dla drzew oraz mrówkowy.
Niniejsza praca opisuje wymienione algorytmy, porównuje je pod kątem wydajności, poprawności oraz czasu działania.

% • W rozdziale 1. Wstęp i cel pracy, autor (autorzy) nakreśla problematykę opisaną lub rozwiązywaną w pracy dyplomowej wraz z uzasadnieniem jej realizacji. Podaje cel i ewentualnie tezę (hipotezę). Syntetycznie opisuje dotychczasowe dokonania w danej tematyce, założenia techniczne oraz może zwięźle przedstawić zawartość poszczególnych rozdziałów. W pracy realizowanej wspólnie przez kilku autorów celowe jest podanie autorów poszczególnych rozdziałów ewentualnie podrozdziałów. Punkty stanowiące element składowy podrozdziału muszą być opracowane przez jednego autora.
% • Nie przytacza się szczegółowych danych liczbowych badanego zagadnienia oraz nie umieszcza się tabel, rysunków czy wykresów. Można odwołać się do ważniejszych pozycji literatury.