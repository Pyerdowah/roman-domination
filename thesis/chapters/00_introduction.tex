\chapter{Wstęp i cel pracy}
Tematem pracy jest analiza algorytmów znajdujących funkcję dominującą rzymską słabospójną w grafach.
Problem znajdywania liczby dominowania rzymskiego słabospójnego jest problemem NP-trudnym. Wersja decyzyjna tego problemu jest NP-zupełna. Nie są znane zatem dokładne algorytmy rozwiązujące problem w czasie wielomianowym. Dodatkową motywacją jest widoczny potencjał zastosowania algorytmu w rozwiązywaniu praktycznych problemów. Dlatego niniejsza praca dokonuje analizy istniejących w literaturze i proponowanych algorytmów rozwiązujących ten problem w sposób zarówno dokładny, jak i przybliżony, w celu znalezienia możliwie skutecznych rozwiązań oraz zastosowań.

\section{Cel pracy}
Celem badawczym pracy jest opisanie istniejących i opracowanie własnych algorytmów znajdujących funkcje dominujące rzymskie słabo spójne, ich analiza oraz porównanie skuteczności i wyciągnięcie wniosków na temat możliwości ich praktycznego zastosowania.

\section{Zakres pracy}
W ramach pracy dokonano systematycznego przeglądu literatury. W literaturze szeroko zdefiniowano ten problem oraz ego złożoność obliczeniową. Zaproponowano również kilka algorytmów rozwiązujących ten problem, zarówno dokładnych - wykorzystujących m.in. programowanie liniowe - jak i przybliżonych. W literaturze zostały również zdefiniowane algorytmy niedokładne, aproksymacyjne, o różnej jakości rozwiązania. Na podstawie znalezionej literatury zaimplementowane zostały dwa algorytmy programowania liniowego oraz $2(1+\epsilon)(1 + \ln(\Delta - 1))$-aproksymacyjny. W ramach własnej pracy, zaimplementowano algorytm dokładny brute force, zachłanny, liniowy dokładny dla drzew oraz mrówkowy. Niniejsza praca opisuje wymienione algorytmy, porównuje je pod kątem wydajności, poprawności oraz czasu działania. Testy przeprowadzono na różnych klasach grafów: gęstych, rzadkich, drzewach oraz bezskalowych. Następnie dokonano analizy potencjalnych praktycznych zastosowań, w których algorytmiczne rozwiązanie tego problemu byłoby przydatne. Na koniec dokonano podsumowania wszystkich wyników dotyczących przeprowadzonych rozważań.