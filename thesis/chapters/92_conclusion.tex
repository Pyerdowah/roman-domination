\chapter{Podsumowanie i wnioski}

\section{Podsumowanie wyników}

Celem badawczym pracy było opisanie istniejących i opracowanie własnych algorytmów znajdujących funkcje dominujące rzymskie słabo spójne, ich analiza oraz porównanie skuteczności i wyciągnięcie wniosków na temat możliwości ich praktycznego zastosowania.\\
\textbf{Udało się zrealizować wszystkie postawione zadania, mianowicie:}
\begin{itemize}
    \item przedstawienie problemu WCRDF oraz jego złożoności obliczeniowej,
    \item przedstawienie algorytmów znajdujących WCRDF oraz algorytmy znajdujące dokładną  $\gamma^{\text{wc}}_R(G)$ oraz przybliżoną, korzystając z różnych technik programowania,
    \item zaimplementowanie i przetestowanie wybranych algorytmów,
    \item wyciągnięcie wniosków na podstawie wyników testów,
    \item dokonanie analizy pod kątem praktycznego zastosowania rozwiązania tego problemu.
\end{itemize}

\textbf{W pracy przetestowano siedem algorytmów:}
\begin{itemize}
    \item algorytmy optymalnie wyznaczające wartość $\gamma^{\text{wc}}_R(G)$: Brute Force, ILP i ILP2 oraz TreeLinear - liniowy dla drzew,
    \item algorytmy przybliżone: zachłanny - Greedy, aproksymacyjny - Approx i mrówkowy - AntColony.
\end{itemize}

Eksperymenty przeprowadzono na grafach gęstych, rzadkich, drzewach oraz bezskalowych. Dodatkowo sprawdzono działanie niektórych algorytmów na grafach rzeczywistych. \\
\textbf{Wyniki wskazują, że:}
\begin{itemize}
    \item Algorytm BruteForce, chociaż prezentuje dokładne, minimalne rozwiązanie problemu, jest najgorzej skalowalny. Już dla kilkunastu wierzchołków grafu, czas działania robi się nierozsądnie długi.
    \item Algorytmy ILP i ILP2 zapewniają najlepsze jakościowo rozwiązania, lecz ich czas działania rośnie znacząco wraz z rozmiarem badanego grafu, co wynika ze złożoności obliczeniowej. Dla grafów gęstych, o dużej liczbie klik, algorytm ILP2 oparty na przepływach faktycznie zaczyna działać szybciej niż ILP, opartym na ograniczeniach cyklicznych.
    \item Algorytm liniowy dla drzew znajduje optymalne rozwiązanie w szybkim czasie i jest skalowalny dla większych instancji drzew. Niestety jego prawidłowe działanie ograniczone jest do jednej klasy grafów.
    \item Algorytm Greedy wyróżnia się ekstremalnie niskim czasem działania oraz zaskakująco dobrą jakością rozwiązań w wielu klasach grafów. Jako jedyny spośród prezentowanych algorytmów jest skalowalny do większych grafów.
    \item Approx zapewnia stabilne i poprawne rozwiązania, z niskim błędem względnym, jednak znacząco pogarsza się dla drzew i struktur bardzo rzadkich. Ponadto udało się zweryfikować, że dla testowanych przykładów grafów udało się pokazać, że wyniki są zgodne z prezentowanym w literaturze współczynnikiem aproksymacji.
    \item AntColony prezentuje najgorszy spośród wszystkich algorytmów czas działania oraz jakość rozwiązania.
\end{itemize}

Ponadto wykazano, że wyniki algorytmów różnią się w zależności od klasy grafu, co sugeruje konieczność doboru metody w zależności od specyfiki problemu.

\section{Wnioski i dalsze kierunki badań}

\textbf{Na podstawie przeprowadzonych badań i analiz sformułowano następujące wnioski:}
\begin{itemize}
    \item WCRDF może stanowić potencjalnie użyteczny model do minimalizacyjnego rozmieszczania zabezpieczeń w sieciach społecznościowych, czy innych naturalnie utworzonych grafach.
    \item Algorytmy dokładne (ILP, ILP2) są dobrym punktem, ale niestety są mało skalowalne.
    \item Algorytm Greedy może być z powodzeniem stosowany w praktyce, szczególnie w dużych instancjach wymagających szybkich przybliżeń.
    \item AntColony w obecnej implementacji nie nadaje się do skutecznego rozwiązywania tego problemu. Dla każdej klasy grafu i prezentował najbardziej niekorzystne wyniki wśród wszystkich testowanych algorytmów.
    \item Zastosowanie WCRDF do problemów praktycznych, czyli np. rozmieszczenia zabezpieczeń w sieci energetycznej czy agentów wykrywających oszustwa w sieci społecznościowej zostało uzasadnione i warte dalszych badań.
\end{itemize}

\textbf{Dalsze kierunki badań mogą obejmować:}
\begin{itemize}
    \item Rozszerzenie testów na inne klasy, jak i na inne wielkości grafów.
    \item Podniesienie jakości wyników algorytmu mrówkowego, między innymi poprzez inną implementację heurystyki lokalnej oraz strategii feromonowej oraz badania w celu znalezienia jak najlepszej heurystyki.
    \item Próba opracowania algorytmów dokładnych, rozwiązywalnych w czasie wielomianowym dla innych klas grafów.
    \item Weryfikacja i przełożenie teoretycznych rozważań na temat praktycznych zastosowań tego problemu na praktyczną analizę i realizację.
\end{itemize}
